\documentclass{mwart}
\usepackage{pdfpages}
\usepackage[utf8]{inputenc}
\usepackage{indentfirst}
\usepackage{polski}
\usepackage{float}
\usepackage{hyperref}
\usepackage{amsmath}
% \usepackage[final]{graphicx}
\usepackage[a4paper, total={6in, 9in}]{geometry}

\usepackage{xcolor}
\usepackage{listings}
\usepackage{longtable}
\usepackage{natbib}
\usepackage{graphicx}
\usepackage{enumitem}
\usepackage{indentfirst}
\usepackage{pgfplots}

\begin{document}

\title{
    \textbf{Przetwarzanie i rozpoznawanie obrazów}\\
    \begin{large}
        Projekt 1
    \end{large}
}

\author{
    Rozwadowski Mikołaj\\
  \texttt{127205}
  \and
    Górka Bartosz\\
  \texttt{127228}
}

\date{}

\maketitle

\section{Charakterystyka systemu}
\label{section:wstep}
System porówywania obrazów został przygotowany na wzór frameworków, gdzie każda z cech liczona jest indywidualnie. W folderze \texttt{features} zawarto miary podobieństw dwóch obrazów (każda z nich może być aktywowana bądź dezaktywowana niezależnie od pozostałych).

Za poprawność działania odpowiada klasa \texttt{Matcher}, w której zastosowano listę cech (tj. pole \texttt{self.features}). Na podstawie sumy wartości podobieństw między obrazami zostaje przygotowana odpowiedź systemu w formacie zgodnym z wymaganiami projektu.

\section{Wstępne przetwarzanie}
\label{section:preprocessing}
Przed rozpoczęciem analizy cech obrazu i próbą porównania następuje proces przygotowania obrazu. Jego realizacja została zaprezentowana w postaci pseudokodu poniżej.

\begin{verbatim}
Wczytaj obraz z wykorzystaniem cv2.imread(filename, 0)
Dokonaj rotacji i przeskalowania obrazu
Wyznacz kontur dzięki funkcji cv2.findContours
Wyznacz polygon - cv2.approxPolyDP
Dokonaj obliczenia charakterystyki udziału koloru białego i czarnego w kolumnach
\end{verbatim}

W przedstawionym powyżej pseudokodzie są dwie metody wymagające głębszej analizy. Pierwsza z nich to operacja rotacji i przeskalowania zaprezentowana w postaci następującego pseudokodu:

\begin{verbatim}
Wyznacz kontury z użyciem cv2.findContours
Z wykorzystaniem cv2.minAreaRect wskaż minimalny prostokąt opisany na figurze
Dokonaj rotacji obrazu dzięki cv2.getRotationMatrix2D
(aplikacja transformacji z użyciem cv2.warpAffine)
Jeżeli centroid czarnego obszaru jest przesunięty w lewo lub prawo {
    Obróć obraz dodatkowo o +/- 90 stopni
}

Przesuń figurę na początek układu współrzędnych
Dokonaj skalowania figury do ustalonych rozmiarów 300px x 200px

Jeżeli liczba białych pikseli do góry obrazu jest większa od tych na dole {
    Oznacza to że podstawa jest do góry
    Dokonaj rotacji o 180 stopni aby znalazła się na dole
}
\end{verbatim}

Drugą z operacji jest liczenie charakterystyki udziału pikseli:
\begin{verbatim}
Podziel przeskalowany już obraz na kolumny o szerokości 1px
Dla każdej z kolumn oblicz ile pikseli ma niezerową wartość
Wynik zapisz na odpowiedniej pozycji w liście charakterystyki obrazu
\end{verbatim}

\section{Charakterystyki}
Jak wskazano w rozdziale \ref{section:wstep}, system pozwala na wykorzystanie wielu miar podobieństw pomiędzy obrazami. Przygotowane implementacje zostaną zaprezentowane w ramach podrozdziałów.

\subsection{Random}
Pierwsza z charakterystyk polegająca na określeniu w sposób pseudolosowy podobieństw pomiędzy obrazami. Nie wykorzystuje w ogóle informacji o obrazach. Powstała do testowania formatu wejścia/wyjścia, ale w ostatecznej wersji projektu nie jest wykorzystywana (waga $0.0$).

\subsection{Liczba wierzchołków}
W podobieństwie tym wykorzystuje się liczbę wierzchołków dla obu obrazów. Kontur wyznaczony w fazie preprocessingu (rozdział \ref{section:preprocessing}) jest wyznaczany z określoną czułością $\epsilon$.

Podobieństwo opisuje się wzorem:
\begin{equation*}
    1 - \frac{|v_1 - v_2|}{\frac{1}{2} \cdot (v_1 + v_2)},
\end{equation*}

gdzie $v_1$ i $v_2$ to liczby wierzchołków w dwóch obrazach.

\subsection{Wariancja udziału koloru białego}
\label{subsection:wariancja-koloru-bialego}
Bazując na wyznaczonych w fazie preprocessingu charakterystykach udziału koloru białego (lista z licznością pikseli w kolumnie) obliczana jest wariancja sumy tych list dla dwóch obrazów. W przypadku drugiego z obrazów, lista jest odwrotnej kolejności. Idea stojąca za metodą jest stosunkowo prosta: w przypadku pasujących do siebie obrazów, wariancja powinna być nieznaczna. Ma to miejsce, kiedy praktycznie każda z pozycji nowou tworzonej listy ma prawie taką samą wartość.

Zastosowano wzór:
\begin{equation*}
    \frac{x}{D^2(w_1 + w_2')},
\end{equation*}

gdzie $x$ to szerokość obrazu, $w_1$ i $w_2$ to wysokości białych poziomych pasków, a $D^2$ oznacza wariancję z próby. 

Takie użycie priorytetuje niską wariancję, a wraz z jej wzrostem maleje wartość podobieństwa.

\subsection{Korelacja udziału koloru białego}
Podobnie jak dla wariancji udziału białego koloru, również w przypadku koleracji bazujemy na wyznaczonej charakterystyce. Tym razem jednakże zastosowano współczynnik korelacji Pearsona.
Dla jednego z obrazów charakterystyka jest dodatkowo podawana w odwrotnej kolejności. Z wyliczonych korelacji wyznacza się wartość bezwzględną i wybiera największą wartość.

\begin{equation*}
\max(|\rho(w_1, w_2)|, |\rho(w_1, w_2')|)
\end{equation*}

\section{Parametry i wagi cech}
Parametry wstępnego przetwarzania i wagi cech dobrano eksperymentalnie, sprawdzając wyniki na zbiorze testowym.
Użyte ostatecznie wartości to:

\begin{itemize}
    \item znormalizowana szerokość: 300,
    \item znormalizowana wysokość: 200,
    \item czułość przybliżenia konturu wielokątem: 6,
    \item wysokość marginesu do wykrywania podstawy: 19,
    \item próg względnego oddalenia centroidu czarnego obszaru od środka: 0.25
\end{itemize}

Wagi cech ustawiono następująco:

\begin{itemize}
    \item wariancja koloru białego: 42\%
    \item liczba wierzchołków: 33\%
    \item korelacja koloru białego: 25\%
\end{itemize}

\section{Wyniki na zbiorach testowych}

Według przyjętej miary oceny, program uzyskał średnio 79,4\% punktów na zbiorach testowych. Łączny czas działania to 13 sekund. Szczegóły przedstawia tabela \ref{table:score}.

\begin{table}
\centering\begin{tabular}{|r|rrrrr|r|}
\hline
zbiór & \#1 & \#2 & \#3 & \#4 & \#5 & wynik \\ \hline
set0  & 6   & 0  & 0 & 0  & 0 & 100.0\% \\
set1  & 20  & 0  & 0 & 0  & 0 & 100.0\% \\
set2  & 20  & 0  & 0 & 0  & 0 & 100.0\% \\
set3  & 20  & 0  & 0 & 0  & 0 & 100.0\% \\
set4  & 20  & 0  & 0 & 0  & 0 & 100.0\% \\
set5  & 182 & 4  & 2 & 3  & 1 & 91.0\%  \\
set6  & 162 & 1  & 2 & 1  & 2 & 81.0\%  \\
set7  & 11  & 4  & 3 & 1  & 0 & 55.0\%  \\
set8  & 40  & 10 & 2 & 6  & 6 & 40.0\%  \\\hline
razem & 481 & 19 & 9 & 11 & 9 & 79.4\% \\
\hline
\end{tabular}
\caption{Wyniki na zbiorach testowych}\label{table:score}
\end{table}

\end{document}
